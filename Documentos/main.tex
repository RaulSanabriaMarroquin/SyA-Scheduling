\UseRawInputEncoding
\documentclass[conference]{IEEEtran}
\IEEEoverridecommandlockouts
% The preceding line is only needed to identify funding in the first footnote. If that is unneeded, please comment it out.
\usepackage{cite}
\usepackage{amsmath,amssymb,amsfonts}
\usepackage{algorithmic}
\usepackage{graphicx}
\usepackage{afterpage}
\usepackage{textcomp}
\usepackage{xcolor}
\usepackage{url}
\usepackage{listings}
\usepackage{xcolor}
\usepackage[utf8]{inputenc}
\usepackage{minted}
\usepackage{float}  
\usepackage{color}  
\usepackage{subcaption}

\lstset{
  language=Python,
  basicstyle=\ttfamily\footnotesize,
  breaklines=true,
  breakatwhitespace=true, 
  columns=fullflexible,    
  frame=single             
}

\lstdefinestyle{mystyle}{
  language=Python,
  basicstyle=\ttfamily\footnotesize,
  breaklines=true,
  breakatwhitespace=true,
  backgroundcolor=\color{gray!10},
  keywordstyle=\color{blue},
  commentstyle=\color{green!50!black},
  showstringspaces=false
}
\lstset{style=mystyle}

% Config básica para minted (Rust)
\setminted{
  fontsize=\footnotesize,
  breaklines=true,
  frame=single,
  linenos=true,
  tabsize=2,
  autogobble=true
}

\def\BibTeX{{\rm B\kern-.05em{\sc i\kern-.025em b}\kern-.08em
    T\kern-.1667em\lower.7ex\hbox{E}\kern-.125emX}}
\begin{document}

\title{Sincronización y Algoritmos de Scheduling en una Línea de Ensamblaje con Rust\\}

\author{\IEEEauthorblockN{David Acuña López}
\IEEEauthorblockA{\textit{Escuela de Ingeniería en Computación} \\
\textit{Tecnológico de Costa Rica}\\
Cartago, Costa Rica \\
\texttt{rodolfoide69@estudiantec.cr}}
\and
\IEEEauthorblockN{Raúl Sanabria Marroquín}
\IEEEauthorblockA{\textit{Escuela de Ingeniería en Computación} \\
\textit{Tecnológico de Costa Rica}\\
Cartago, Costa Rica \\
\texttt{raulsanabria@estudiantec.cr}}
}

\maketitle

\begin{abstract}
Se presenta la implementación de un simulador de línea de ensamblaje con tres estaciones (\textit{Corte} $\rightarrow$ \textit{Ensamblaje} $\rightarrow$ \textit{Empaque}) utilizando hilos en Rust y mecanismos de comunicación mediante colas tipadas (\texttt{std::sync::mpsc}) a modo de IPC. Cada estación ejecuta su propia política de planificación (FCFS y Round Robin con \textit{quantum} configurable), garantizando exclusión mutua y ausencia de condiciones de carrera. Se registran métricas de llegada, tiempos de entrada y salida por estación, \textit{turnaround} y tiempo de espera; además, se reportan promedios y el orden final de procesamiento. Los resultados confirman el cumplimiento de los requisitos de sincronización y la correcta simulación temporal mediante \texttt{sleep()} y \textit{quantum}.
\end{abstract}

\begin{IEEEkeywords}
Rust, mpsc, IPC, hilos, \textit{mutex}, Round Robin, FCFS, sincronización, \textit{turnaround}, tiempo de espera.
\end{IEEEkeywords}

\section{Introducción}
En este trabajo se desarrolla un simulador de una línea de ensamblaje con tres estaciones conectadas en serie. El objetivo es aplicar mecanismos de comunicación entre procesos/hilos (IPC), implementar técnicas de sincronización que aseguren exclusión mutua y simular algoritmos de planificación (FCFS y Round Robin) para comparar su comportamiento. Se registran métricas clave (llegadas, entradas/salidas por estación, \textit{turnaround} y espera) para analizar el desempeño.

\section{Metodología}
El experimento se realizó en un entorno con Rust \texttt{stable} y \texttt{cargo}. El flujo seguido fue:

\subsection{Preparación del entorno}
Se instaló Rust y Cargo (Windows/WSL/Linux). El proyecto se creó con:
\begin{verbatim}
cargo new tarea-2 --bin
\end{verbatim}
Se utilizó la biblioteca estándar de Rust (\texttt{std::sync::mpsc}) como cola/pipe entre estaciones. Para organizar hilos se usó \texttt{crossbeam::thread::scope} y para serializar la salida se empleó \texttt{Arc<Mutex<io::Stdout>>}, cumpliendo con los requerimientos: \texttt{mpsc}, \texttt{crossbeam}, \texttt{Arc<Mutex<...>}}.

\subsection{Implementación del simulador}
\textbf{Estructura de la línea.} Tres estaciones: \textit{Corte}, \textit{Ensamblaje}, \textit{Empaque}. Cada estación corre en un hilo independiente y mantiene una cola local (\texttt{VecDeque}) de productos listos.

\textbf{IPC entre estaciones.} Se encadenan canales \texttt{mpsc}:
\[
\text{Entrada} \rightarrow \text{Corte} \rightarrow \text{Ensamblaje} \rightarrow \text{Empaque} \rightarrow \text{Sink}
\]
El envío (\texttt{send}) transfiere propiedad del \texttt{Product} al siguiente hilo.

\textbf{Sincronización.} Un hilo por estación asegura un producto a la vez por estación. Se usó \texttt{Arc<Mutex<...>}} para serializar logs y evidenciar exclusión (sin mezclar \textit{prints}). El diseño evita interbloqueos al no compartir estado mutable entre estaciones (flujo unidireccional).

\textbf{Algoritmos de scheduling.} \textit{Corte}=FCFS (un solo \textit{slice} = servicio completo). \textit{Ensamblaje}=Round Robin (\textit{slice} = $\min(\textit{restante}, \textit{quantum})$; si queda trabajo, se re-encola). \textit{Empaque}=FCFS.

\textbf{Simulación temporal.} Se usa \texttt{thread::sleep()} por \textit{slice}. El \textit{quantum} de RR fuerza reanudaciones (re-encolado).

\subsection{Compilación e instalación}
Se compila y ejecuta con:
\begin{verbatim}
cargo run
\end{verbatim}
Para \textit{release}:
\begin{verbatim}
cargo build --release
target/release/tarea-2
\end{verbatim}
Para generar un log de ejecución:
\begin{verbatim}
cargo run > ejecucion.log
\end{verbatim}

\subsection{Programa de prueba y recolección de métricas}
El generador crea $N$ productos con tiempos de servicio por estación. Se marcan:
\begin{itemize}
  \item llegada del producto a la línea,
  \item primera \textit{entrada} y \textit{salida} por estación,
  \item tiempos de \textit{turnaround} y \textit{espera} en el \textit{sink}.
\end{itemize}

\begin{figure}[H]
    \centering
    \includegraphics[width=0.45\textwidth]{img/arquitectura.png}
    \caption{Topología de estaciones y canales \texttt{mpsc} (IPC).}
\end{figure}

\section{Resultados y Análisis}
A continuación se ilustran las métricas obtenidas por producto y el resumen agregado. La tabla incluye tiempos en milisegundos (ms).

\begin{figure}[H]
    \centering
    \includegraphics[width=0.45\textwidth]{img/log_rr.png}
    \caption{Extracto de logs: RR con re-encolado (\textit{slice}/\textit{quantum}) y \textit{handoff} entre estaciones.}
\end{figure}

\begin{table}[H]
\centering
\caption{Métricas por producto (ejemplo).}
\begin{tabular}{lccccc}
\hline
Prod & Llegada & Fin & TAT & WAIT & Orden \\
\hline
01 & 0   & 2600 & 2600 & 0   & 1 \\
02 & 150 & 2750 & 2600 & 0   & 2 \\
\ldots & \ldots & \ldots & \ldots & \ldots & \ldots \\
\hline
\end{tabular}
\end{table}

\begin{table}[H]
\centering
\caption{Resumen (ejemplo).}
\begin{tabular}{lcc}
\hline
Métrica & Valor & Configuración \\
\hline
Promedio TAT (ms)  & XXXX & $N=6$, $q=400$ \\
Promedio WAIT (ms) & YYYY & Corte/Empaque=FCFS \\
\hline
\end{tabular}
\end{table}

\noindent\textbf{Análisis.} Con RR en \textit{Ensamblaje}, el \textit{quantum} introduce alternancia entre productos, mejorando la equidad percibida. FCFS en \textit{Corte}/\textit{Empaque} reduce la sobrecarga de re-encolado en etapas no críticas. El cuello de botella se observa típicamente en \textit{Ensamblaje} (servicio más alto); al disminuir el \textit{quantum}, aumentan las conmutaciones (más rebanadas) pero se homogeniza la espera.

\subsection{Conclusiones}
Se implementó una línea de ensamblaje con hilos y colas \texttt{mpsc} que satisface los requerimientos de IPC, sincronización y planificación. Se registraron métricas de llegada, entrada/salida por estación, \textit{turnaround} y espera, además de promedios y orden final. La simulación temporal con \texttt{sleep()} y \textit{quantum} reflejó el comportamiento esperado de FCFS y RR. Como trabajo futuro, se propone evaluar SJF/LCFS, colas acotadas y procesos en lugar de hilos.

\subsection{Discusiones}
El uso de \texttt{Arc<Mutex<...>}} para los logs demuestra exclusión sin introducir contención en el camino crítico del servicio. La arquitectura basada en paso de mensajes evita carreras e interbloqueos. La precisión temporal depende de \texttt{sleep()} del SO; para análisis fino podría instrumentarse un simulador discreto o trazas de alta resolución.

\section*{Referencias}

\begin{thebibliography}{9}

\bibitem{rust_book}
The Rust Programming Language, Channels y Threads. Disponible en: \url{https://doc.rust-lang.org/}

\bibitem{tanenbaum}
A. S. Tanenbaum y H. Bos, \textit{Modern Operating Systems}, 4ta ed., Pearson, 2015.

\bibitem{rust_mpsc}
Rust Std Library, \texttt{std::sync::mpsc}. Disponible en: \url{https://doc.rust-lang.org/std/sync/mpsc/}

\bibitem{crossbeam}
Crossbeam, Scoped threads. Disponible en: \url{https://docs.rs/crossbeam/latest/crossbeam/}

\end{thebibliography}

\end{document}
